\documentclass[12pt]{article}
\usepackage{amssymb,amsmath}
\usepackage[pdftex]{graphicx}
\usepackage{epsfig,subfigure}
\usepackage{epstopdf}
\usepackage{bm,url}
\DeclareGraphicsExtensions{.jpg,.pdf,.png,.eps,.ps}
\usepackage[usenames, dvipsnames]{color}
\usepackage{ulem}
\usepackage{xcolor}
\usepackage{soul}
\usepackage{hyperref}

\textheight = 522pt
\textwidth = 450pt
\oddsidemargin 0.0in

\newcommand{\comment}[1]{\textcolor{Green}{(Comment: #1)}}
\newcommand{\todo}[1]{\textcolor{red}{(TODO: #1)}}
\newcommand{\rev}[1]{\textcolor{black}{#1}}



\newcommand{\exec}{{Executive Team}}
\newcommand{\shorte}{{ET }}  %abbrev for \exec

%% Define a new 'leo' style for the package that will use a smaller font.
\makeatletter
\def\url@leostyle{%
  \@ifundefined{selectfont}{\def\UrlFont{\sf}}{\def\UrlFont{\small\ttfamily}}}
\makeatother
%% Now actually use the newly defined style.
\urlstyle{leo}

\title{CMB-S4 Bylaws CoC}
\author{CMB-S4 GB}
\date{January 27, 2022}

\begin{document}

\maketitle

                                                           

\section{Principles and Responsibilities}
CMB-S4 is committed to creating and upholding a safe, inclusive, welcoming, and open community where everyone can grow and reach their full potential. CMB-S4 is a large, complex, and international collaboration with members from diverse backgrounds and institutions. We value the rich perspectives and experiences that arise from diversity within our community, and aim to empower individuals at early stages in their careers and those with underrepresented identities to contribute fully and to feel welcomed and supported. Members of the CMB-S4 collaboration are expected to conduct themselves with scientific integrity, uphold the highest ethical standards, communicate openly and respectfully, and further justice, equity, diversity, and inclusion in all collaboration interactions. To further these goals, members of our community have a responsibility to remove barriers within our collaboration, including those caused by social injustice, inequity, and discrimination based on a person's identity.

The conduct of collaboration members is guided by the following principles:  

\begin{itemize}
    

\item Respect, support, equity, and fairness in interactions with others

\item Excellence, integrity, and honesty in all aspects of research, including the dissemination of results

\item Safe, supportive, and welcoming work environments free from discrimination, bullying, and harassment  

\item Developing a culture that promotes justice, equity, diversity, and inclusion

\item Freedom to responsibly pursue science without interference or coercion

\item Unselfish cooperation in research

\end{itemize}


Collaboration members have the following responsibilities:

\medskip

{\it Integrity:} Members will act with honesty in the interest of the advancement of science and treat others with respect, equity, and fairness. 

{\it Adherence to Law and Regulations:} Members will adhere to laws and regulations related to the professional conduct of research, to this code of conduct, and to written CMB-S4 policies, including those for safety, publications, elections, and membership.

{\it Fostering a Supportive Environment:} CMB-S4 members are responsible for creating and upholding a safe, open, supportive, and welcoming environment for learning, conducting, and communicating science with integrity, respect, fairness, and transparency at all organizational levels and in all scientific endeavors. Members should acknowledge the privilege and power they hold in different spaces and use it to empower and amplify voices that are being excluded, work to ensure that all colleagues have equal access to opportunities, and be receptive to discussions on how to improve communication and working relationships. Members should also take the initiative to educate themselves on and work to remove challenges or barriers to success that colleagues may face, challenge their assumptions, and intervene when others are exhibiting conduct unbecoming of a CMB-S4 member.

{\it Scientific responsibility:} Members are responsible for the integrity of their contributions to all publications, funding applications, reports, and other representations of their research. When reviewing the work of CMB-S4 colleagues (e.g., in working group telecons, internal project reviews or publication reviews), they will provide fair, constructive, impartial, and rigorous evaluations.  Members will be receptive to constructive criticism and responsive to review. 

{\it Responsibility to the future:} More senior CMB-S4 members will strive to help advance the careers of more junior members where possible, including by ensuring fair access to information, data, leadership roles, and opportunities to give external talks. This includes awareness of the power dynamic between junior and senior members, and appropriate interactions in light thereof.

%{\it Misconduct:} CMB-S4 members will not engage in discrimination, harassment, bullying, dishonesty, fraud, misrepresentation, coercive manipulation, censorship, or other misconduct that alters the content, veracity, or meaning of research findings or that may affect the planning, conduct, reporting, or application of science. This applies to all professional, research, and learning environments. 

{\it Reporting Irresponsible Research Practices and Misconduct:}  Members should report suspected misconduct affecting themselves or others, and be supportive of those who report.  Reporting procedures are specified below.\\


Some examples of behaviors in violation of our code of conduct include, but are not limited to:

\begin{itemize}

\item Hate speech and discrimination. %We do not tolerate speech that attacks a person or group of people on the basis of their identities.

\item Microaggressions. %While microaggressions can be unintentional, we must acknowledge the harm that they cause and encourage growth and learning from mistakes. \todo{Definition and/or examples}

\item Misrepresentation of authorship of work.

\item Bullying and harassment. %We do not tolerate bullying or harassment. This means any habitual badgering or intimidation targeted at a specific person or group of people. In general, if you are told that your actions targeted at a person/group are unwanted and violate our agreed values, yet you continue to engage in them, then you are bullying and/or harassing the community.

\item Sexual harassment and unwanted advances. %Unwanted advances/sexual harassment are not tolerated and include instances that occur in ``after hours” spaces like conference dinners during meetings.

\item Retaliatory behavior against those who report incidents.% is not tolerated in any form. %Any reports of Code of Conduct violations and the information provided in those reports will not be used against the reporter. All offenses are investigated and reports of Code of Conduct violations made in bad faith will be dismissed.}
\end{itemize}

The code of conduct applies to all interactions involving CMB-S4 members, including both direct work environments and ``after hours'' spaces like conference dinners during meetings.





\section{Resolution Process}

The Governing Board is responsible for ensuring that issues concerning possible violations of this Code of Conduct are resolved in a prompt and timely manner. The Governing Board is also responsible for resolving other issues related to conduct not explicitly specified in this document.  Maximum confidentiality will be maintained consistent with the nature of the complaint, its proposed resolution, and legal requirements.  Where legal requirements apply, the officers of the CMB-S4 collaboration will work with appropriate authorities as required to resolve the allegation.  However, CMB-S4 investigations and any resulting sanctions are independent of any external disciplinary proceedings or legal actions that may arise on related issues, though the results of such proceedings may inform the Governing Board’s deliberations where appropriate.


\textbf{Definition}: In all discussions in this section a ``Conflict of Interest'' is found for those persons (1) at the same institution as a party involved in a misconduct allegation, (2) related (by family or law) to a party in the same, or (3) for anyone who self-declares such a conflict. 


\subsection{Reporting}

Concerns about or allegations of violations of the Collaboration Code of Conduct or other inappropriate behavior should be reported in writing. Collaboration members with such concerns may either address one of the CMB-S4 Ombudspersons (see website) or report directly to one of the collaboration officers listed below. The Ombudspersons can discuss concerns or allegations in complete confidence, and may be able to help resolve them. Discussions with Ombudspersons do not constitute formal reports, and the Ombuspersons do not have any disciplinary authority; the Ombudspersons, however, can provide information and answer questions about the process for formal complaints. Formal complaints should be reported in writing to the Collaboration Spokespersons, the Chair of the Governing Board, or any member of the Governing Board, in that order of preference, with the choice left to the reportee (contact information is in Section~\ref{sec:contacts}).


The recipient of the initial report will promptly (where possible, within one working day) inform the Spokespersons and the Chair of the Governing Board (except in cases where any of these officers of the collaboration are known to have a conflict of interest). Within 14 business days of receiving an allegation or report of concern, this group will conduct an initial assessment to determine if further investigation is warranted. \rev{The Ombudsperson, Spokespeople, and Governing Board are required to receive training on how to handle complaints following collaboration policy.} 
%\todo{Add some information about Ombudsperson and Spokespeople/GB training about handling reports.}



\subsection{Investigation and Resolution}

If it is decided that a complaint or allegation requires further investigation, the Spokespersons and the Chair of the Governing Board will form an investigating committee composed of no fewer than three members of the Governing Board within 5 business days. In cases of conflict of interest, the Spokesperson(s) or Chair of the Governing Board will identify a different member of the Governing Board to take the place of the conflicted individual in selecting the investigating committee. Parties to the complaint will be informed of the membership of the investigating committee. If any party has concerns about the composition of the committee, they should promptly inform the Governing Board, who may choose to take action by adjusting the committee membership.

The investigating committee will then investigate. The person under investigation will be given a chance to respond to the allegations.  The investigating committee may work with the parties to devise a satisfactory resolution to the issue.  Accused parties will be presumed innocent unless found otherwise by a preponderance of the evidence. The Governing Board will \rev{monitor the progress of} the investigatory committee to ensure a timely resolution. At the conclusion of the investigation, the investigating committee will present its findings and recommendations, including possible resolutions, to a meeting of the Governing Board, with Spokespersons in attendance. 

Based on these findings, the Governing Board may, by majority vote, dismiss the allegation, approve the proposed resolution, or implement sanctions.  Sanctions may include, but are not limited to, appropriate corrective actions or training, censure of the accused party, prohibition from specified collaboration actions for a period of time, or, by supermajority vote, removal of the accused party from collaboration leadership positions or expulsion from the collaboration. Following the decision by the Governing Board, if substantial new evidence becomes available, the Governing Board is empowered to reopen the investigation.
 
The Spokespersons and GB Chair will report on the status of all active cases at each meeting of the Governing Board.

\subsection{Records}

Records of complaints and the resolution process will be kept in both private and public forms. Public record-keeping, with identifying information removed, promotes awareness of this policy and issues of concern to collaboration members. Reporting complaints and actions taken encourages collaboration members to act in ways consistent with the collaboration's goals and encourages reporting in the event of behavior contrary to those goals. Private record-keeping with fuller information maintains the governing board's interest in its history and identifying repeated situations of concern.

The private record will include the full complaint and actions of the investigatory committee. Private records will be maintained by the chair of the Governing Board. These will be kept confidential to the extent permitted by law and institutional policies, and may be viewed by the Governing Board in future only when directly related to an ongoing CoC investigation. In the public form of the complaint, identifying information will be removed in a manner consistent with the wishes of both the complainant and respondent, with a suggested redaction proposed by the investigatory committee. Disagreements between those parties about the contents of the public report will be resolved by the Governing Board, which has final discretion over the contents of public records.

%\todo{We need to acknowledge AGU, APS, and AAS COC documents, on which parts of Section 1 are based.}  % And make sure we aren't plagiarizing them!

\subsection{Contacts}\label{sec:contacts}
\rev{The current listing of all the individual Governing Board members and the Chairs can be found at:\\
\url{https://cmb-s4.atlassian.net/wiki/spaces/XC/pages/15597586/Governing+Board}.\\
You may contact any member of the Governing Board, Spokespeople, or Ombudspeople individually or through the lists below:\\
\begin{itemize}
    \item CMB-S4 Spokespeople: \url{spokespeople@cmb-s4.org}
    \item Ombudspeople: \url{ombudspeople@cmb-s4.org}
    \item Full Governing Board: \url{gb@cmb-s4.org}
\end{itemize}}
\rev{Email addresses for all CMB-S4 members can be found at \url{https://people.cmb-s4.org/public/showdir.php}.}

\subsection{Acknowledgements}
\rev{Parts of this code of conduct were based on existing codes of conduct from the AGU, APS, APS DPF, Simons Observatory, and the AAS.} % Make sure we aren't plagiarizing them and make sure we!


\end{document}
