\documentclass[12pt]{article}
\usepackage{amssymb,amsmath}
\usepackage[pdftex]{graphicx}
\usepackage{epsfig,subfigure}
\usepackage{epstopdf}
\usepackage{bm,url}
\DeclareGraphicsExtensions{.jpg,.pdf,.png,.eps,.ps}
\usepackage[usenames, dvipsnames]{color}
\usepackage{ulem}

\textheight = 522pt
\textwidth = 450pt
\oddsidemargin 0.0in

\newcommand{\Comment}[1]{\textcolor{Blue}{(Comment: #1)}}

\newcommand{\exec}{{Executive Team}}
\newcommand{\shorte}{{ET }}  %abbrev for \exec

%% Define a new 'leo' style for the package that will use a smaller font.
\makeatletter
\def\url@leostyle{%
  \@ifundefined{selectfont}{\def\UrlFont{\sf}}{\def\UrlFont{\small\ttfamily}}}
\makeatother
%% Now actually use the newly defined style.
\urlstyle{leo}

\begin{document}



\section*{Proposed lists of Key Science Topics}
%\Comment{[Ed. note: below are various suggestions for topics that could be considered ``Key'' papers and given alphabetical author lists.  They are organized from narrow to broad scope, where the broadest scope includes all science with CMB-S4.  \textbf{These should be a separate document maintained by the Science Council and not in the bylaws.}]}

\textbf{Super narrow scope}

\noindent\textit{Idea is ``achieved S4 science target from Science book or detection''}

Overview Papers

Data release of maps

B-mode detection or achieved S4 science target

Neutrino mass detection or achieved S4 science target

Neff detection or achieved S4 science target

Dark matter detection or achieved S4 science target 

New highly-significant discovery, such as a new planet

\vskip 10pt
\noindent\textbf{Narrow scope} 

\noindent\textit{Idea is ``new measurements from Science book''}

Overview Papers

Data release of maps

B-mode limits (e.g. updated annually) or detection

Neutrino mass limits or detection

Neff limits or detection

Dark matter limits or detection (axion isocurvature, etc.)

New highly-significant discovery, such as a new planet

\vskip 10pt
\noindent\textbf{Broad scope}

\noindent\textit{Idea is ``all data products''}

Overview Papers

Data release of maps

T/E/B Power Spectra (including B mode limits/detection)

Lensing Potential Map and Power spectra

Release of Likelihood

Cosmological Parameters (including all neutrino mass, Neff, dark matter limits/detections)

Galaxy Cluster Catalog

Point Source Catalog

\vskip 10pt
\noindent\textbf{Broadest scope}

\noindent\textit{Idea is ``all data products and all science from the sky''}

Key Science topics could include almost everything except technical papers and lab-characterizations of equipment


\end{document}
