%\section{Collaboration and Institution Membership}

\textcolor{red}{To be determined by Membership working group}


Updated:  2/15/2018

\section{CMB-S4 Collaboration Membership Policy}

The CMB-S4 collaboration consists of Ph.D. scientists, engineers, Ph.D. thesis students, undergraduate students, and others who contribute significantly to the CMB-S4 program. Membership conveys certain rights as described below, but comes with the obligation of an ongoing commitment of a substantial fraction of members' research time to the CMB-S4 program.

\vspace{0.2in}
\noindent


\subsection{Membership Council}
\begin{itemize}

\item 

\item The duties of the Membership Council are to review and evaluate membership applications, review annual activity reports, and recommend changes in membership status. 
\end{itemize}

\subsection{Membership Types}

\begin{itemize}

\item {\bf Senior Member:} A Senior Member of the collaboration is a member who has a permanent appointment or an appointment, that under normal circumstances can be expected to be renewed indefinitely.   This includes tenure-track appointments at universities and their equivalents elsewhere.  

\item {\bf Postdoctoral Member:} A postdoc working with a Senior Member at their institution can be designated as a Postdoctoral Member by that Senior Member.  Postdocs that reside at an institution where there is no Senior Member can apply to become a Postdoctoral Member and have their application evaluated by the Membership Council on a case-by-case basis. 

\item {\bf Student Member:} A graduate student working with a Senior Member at their institution can be designated as a Student Member by that Senior Member.  We do not anticipate granting membership to students who are not supervised by a Senior Member.

\item {\bf Provisional Member:}  A Provisional Member is a potential senior member who has not yet been approved for senior member status.  This is intended to be a temporary status allowing the member to demonstrate constructive engagement with the collaboration at a level that qualifies them for Senior Membership.

\item {\bf Legacy Member:}  A Legacy Member is a former member who contributed in a key manner to the project infrastructure, but is no longer engaged with the collaboration and is therefore no longer a member.  This status is intended to convey authorship rights to such former members, and to bypass the normal membership procedures should they wish to re-engage with the collaboration.

\end{itemize}



\subsection{Membership Rights}

\begin{itemize}
\item Herein, ``members'' refers to Senior, Postdoctoral, Student and Provisional members (but not Legacy Members) unless otherwise qualified.

\item Senior and Postdoctoral Members vote for representation on the Governing Board.  

\item Members have full data access, including during the proprietary period for data that are eventually released.

\item Members (including Legacy Members) have the right to be listed as a co-author on CMB-S4 publications as specified in the Publication Policy.

\item Members have access to computational resources designated for CMB-S4, according to the policies of the relevant computational resources working group.

\end{itemize}

\subsection{Membership Requirements}
\begin{itemize}

\item Members must commit effort to approved infrastructure tasks, which can include, for example, designing, building, and testing software, hardware, or simulations, as well as management. 


\end{itemize}

\subsection{Membership Application and Approval Process}
After the CMB-S4 collaboration is formed with a first set of initial members (see section 1.6), the process for membership to CMB-S4 will be as specified below.

\begin{itemize}

\item Potential Senior Members will apply for Provisonal Membership via a written application where they specify their proposed work on CMB-S4. 

\item Independent postdocs not co-located with a Senior Member can apply for Postdoctoral Membership and have their application reviewed on a case-by-case basis. 

\item Postdoctoral members can apply for ``Intended Senior Membership Status", which would convey that the postdoc will have Senior Membership status when moving to a permanent appointment or an appointment that, under normal circumstances, can be expected to be renewed indefinitely.  The Membership Council will decide the requirements for achieving this status, the achievement of which grants Senior Membership at the new institution.

\item Applications are reviewed by the Membership Council. The Membership Council recommends membership to the Governing Board.  The Governing Board approves membership.

\end{itemize}

\subsection{Initial Membership}
The CMB-S4 collaboration will initially consist of members who have satisfied the following criteria:

\begin{itemize}
\item Attended at least two CMB-S4 collaboration meetings, and 

\item Voted on the governance bylaws.
\end{itemize}

Within the first year of the Collaboration's establishment, prospective members can be granted membership by successfully petitioning the Membership Council despite not satisfying the above criteria.  The Membership Council may approve such applications during the first year without action by the Governing Board.  

\subsection{Membership Review and Changes in Status}
\begin{itemize} 

\item Each Senior Member, and each independent postdoc not co-located with a Senior Member, will submit an annual activity report to the Membershp Council.  Senior members can discuss activities of their supervisees (e.g. postdocs and students) in their report. The Membership Council will review those reports, consulting with collaboration members and working group leaders as appropriate.

\item Provisional members will submit a annual activity report to the Membership Council, which will review the report and determine whether the provisional member should be promoted to a Senior Member status, continue as a Provisional Member, or have their membership revoked.  Provisional Membership is intended to be a temporary status.

\item Members leaving the collaboration may be granted Legacy Membership upon review by the Membership Council and approval by the Govering Board.

\item If the effort of any member over the previous year seems lacking, the Membership Council will bring this to the attention of the Co-spokepersons and the Governing Board.  This can also be done at any time, should the Membership Council deem the actions of a member to have been egregious and detrimental to CMB-S4.

\item The Governing Board has the authority to grant or terminate all forms of membership.

\item When a person's membership in the collaboration is terminated, they will no longer have access to the CMB-S4 document and database repository, internal forums, computing resources and data.  Authorship rights will continue, or not, as prescribed in the Publication Policy bylaws.


\end{itemize}

\subsection{Procedure to Change Membership Policies}

The Governing Board may change Membership policies by a 2/3 majority vote of voting members.

