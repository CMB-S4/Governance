%\section{Collaboration and Institution Membership}

\textcolor{red}{To be determined by Membership working group}


Updated:  2/9/2018

\section{CMB-S4 Collaboration Membership Policy}

The CMB-S4 collaboration consists of Ph.D. scientists, engineers, Ph.D. thesis students, undergraduate students, and others who contribute significantly to the CMB-S4 program. Membership conveys certain rights as described below, but comes with the obligation of an ongoing commitment of a substantial fraction of members' research time to the CMB-S4 program.

\vspace{0.2in}
\noindent
\textcolor{blue}{Note: comments given in blue text are not part of the official draft, and will be removed for the next iteration. }

\subsection{Membership Committee}
\begin{itemize}

\item 
\textcolor{blue}{Composition and selection:  The composition of and method of selecting members of the Membership Committee should be given in the Governance Policy section.  We recommend about 6 to 8 people on this committee, sized to be large enough to get the work done but small enough that everyone feels essential.  We note that this committee will be a lot of work and should count as significant service to CMB-S4.  This body should probably be elected, at least in part, in a fashion similar to the Governing Board.  We should ensure to have representatives from instrumentation and analysis working groups.  This committee should probably consist of Senior Members only, but care should be taken to ensure its decisions are transparent.}

\item The duties of the Membership Committee are to review and evaluate membership applications, review annual activity reports, and recommend changes in membership status. 
\end{itemize}

\subsection{Membership Types}

\begin{itemize}

%\item \editorial{Needing Further Discussion: Are there member institutions, and if so what does that mean?  It will presumably flow down to different levels of membership if so.}  \textcolor{blue}{Membership Working Group (MWG) default: "no" for now, but this could be changed in the future, especially if institutions are asked to provide significant funding or in-kind resources.}

\item {\bf Senior Member:} A Senior Member of the collaboration is a member who has a permanent appointment or an appointment, that under normal circumstances can be expected to be renewed indefinitely.   This includes tenure-track appointments at universities and their equivalents elsewhere.  

\item {\bf Postdoctoral Member:} A postdoc working with a Senior Member at their institution can be designated as a Postdoctoral Member by that Senior Member.  Postdocs that reside at an institution where there is no Senior Member can apply to become a Postdoctoral Member and have their application evaluated by the Membership Committee on a case-by-case basis. 

\item {\bf Student Member:} A graduate student working with a Senior Member at their institution can be designated as a Student Member by that Senior Member.  We do not anticipate granting membership to students who are not supervised by a Senior Member.

\item {\bf Provisional Member:}  A Provisional Member is a potential senior member who has not yet been approved for senior member status.  This is intended to be a temporary status allowing the member to demonstrate constructive engagement with the collaboration at a level that qualifies them for Senior Membership.

\item {\bf Legacy Member:}  A Legacy Member is a former member who contributed in a key manner to the project infrastructure, but is no longer engaged with the collaboration and is therefore no longer a member.  This status is intended to convey authorship rights to such former members, and to bypass the normal membership procedures should they wish to re-engage with the collaboration.

\end{itemize}

%\subsection{Builder Status}
%\editorial{We are trying to eliminate the broad-brush of ``builder status'' by defining legacy members, and spelling out postdoc transitions to senior membership, below.  The current draft of the publication policy appears to treat builders and members equally;  that document should consider legacy members.}
% \begin{itemize}
% \item \editorial{Needing Further Discussion: Is there a category called ``Builder Status'' that members can achieve?"} \textcolor{blue}{The MWG is flexible on this and the metric for achieving this;  below is only provided as a strawperson.}

% \item Members can achieve ``Builder Status'' by completing three years of project work at greater than 50\% available research time effort, post PhD, toward approved infrastructure tasks.  These infrastructure tasks can include, for example, designing, building, and testing software, hardware, or simulations, as well as management. 

% \item Progress toward builder status will be quantified on annual reports reviewed by the MWG. \editorial{Needing Further Discussion: If postdocs are not formally reviewed, how do they accumulate effort toward builder status?}

% \item Approval of builder status will be made by the deciding board, upon recommendation by the MWG.

% \item \editorial{Needing Further Discussion: What are the rights of people achieving ``Builder Status'' in terms of publications and data access?  Are there additional rights?}  \textcolor{blue}{MWG default start: Builder status conveys the right to be listed as a co-author on CMB-S4 publications as specified in the Publication Policy, regardless of current membership status.}

% \end{itemize}


\subsection{Membership Rights}

\begin{itemize}
\item Herein, ``members'' refers to Senior, Postdoctoral, Student and Provisional members (but not Legacy Members) unless otherwise qualified.

\item Senior and Postdoctoral Members vote for representation on the Governing Board.  \textcolor{blue}{(Details left to Governance Policy document.)}

\item Members have full data access, including during the proprietary period for data that are eventually released.

\item Members (including Legacy Members) have the right to be listed as a co-author on CMB-S4 publications as specified in the Publication Policy.

\item Members have access to computational resources designated for CMB-S4, according to the policies of the relevant computational resources working group.

\end{itemize}

\subsection{Membership Requirements}
\begin{itemize}

% \item \editorial{Needing Further Discussion: Are membership requirements in the near term different from what they will be later as the project evolves, eg when there is a funded project?  How will initial membership be decided?} \textcolor{blue}{MWG default: Coordinate with Governance working group}

% \item \editorial{Needing Further Discussion: Do we require any kind of institutional buy-in, cash or in-kind?} \textcolor{blue}{MWG default: given the barriers to this at some likely member institutions, currently "no", and moving to such should only be done if there is a clear understanding of what this is needed for.}

\item Members must commit effort to approved infrastructure tasks, which can include, for example, designing, building, and testing software, hardware, or simulations, as well as management. 
\textcolor{blue}{(We are leaving this vague on purpose, anticipating that this will only get serious - in terms of approving tasks and quantifying required effort - when a project actually exists.)}  
% \textcolor{blue}{MWG default: once the project is up and running one might say X\% of available research time per year, but not clear how that works for the next few years.  This include mentoring and advising time of graduate students and postdocs working on CMB-S4.}

\end{itemize}

\subsection{Membership Application and Approval Process}
After the CMB-S4 collaboration is formed with a first set of initial members (see section 1.6), the process for membership to CMB-S4 will be as specified below.

\begin{itemize}

\item Potential Senior Members will apply for Provisonal Membership via a written application where they specify their proposed work on CMB-S4. 

\item Independent postdocs not co-located with a Senior Member can apply for Postdoctoral Membership and have their application reviewed on a case-by-case basis. \textcolor{blue}{We believe that such independent postdocs should in practice be assigned a contact/mentor, but have not stipulated such in this policy document.}  

\item Postdoctoral members can apply for ``Intended Senior Membership Status", which would convey that the postdoc will have Senior Membership status when moving to a permanent appointment or an appointment that, under normal circumstances, can be expected to be renewed indefinitely.  The Membership Committee will decide the requirements for achieving this status, the achievement of which grants Senior Membership at the new institution.

\item Applications are reviewed by the Membership Committee. The Membership Committee recommends membership to the Governing Board.  The Governing Board approves membership.

\end{itemize}

\subsection{Initial Membership}
The CMB-S4 collaboration will initially consist of members who have satisfied the following criteria:

\begin{itemize}
\item Attended two CMB-S4 collaboration meetings 

\item Voted on the governing bylaws
\end{itemize}

In addition, people can be granted initial membership by 
successfully petitioning the Membership Committee despite not satisfying the above criteria.  


\subsection{Membership Review and Changes in Status}
\begin{itemize} 

\item Each Senior Member, and each independent postdoc not co-located with a Senior Member, will submit an annual activity report to the Membershp Committee.  Senior members can discuss activities of their supervisees (e.g. postdocs and students) in their report. The Membership Committee will review those reports, consulting with collaboration members and working group leaders as appropriate.

\item Provisional members will submit a annual activity report to the Membership Committee, which will review the report and determine whether the provisional member should be promoted to a Senior Member status, continue as a Provisional Member, or have their membership revoked.  Provisional Membership is intended to be a temporary status.

\item Members leaving the collaboration may be granted Legacy Membership upon review by the Membership Committee and approval by the Govering Board.

\item If the effort of any member over the previous year seems lacking, the Membership Committee will bring this to the attention of the Co-spokepersons and the Governing Board.  This can also be done at any time, should the Membership Committee deem the actions of a member to have been egregious and detrimental to CMB-S4.

\item The Governing Board has the authority to grant or terminate all forms of membership.
%\editorial{Needing Further Discussion: What is official path to terminating membership?  Do we terminate membership of someone who has done a lot of infrastructure work in the past, but currently is inactive?  How does this relate to ``Builder status'' (see above)?}  The Deciding Board, in consultation with the MWG, has the authority to revoke membership, and membership rights, from any current member.

\item When a person's membership in the collaboration is terminated, they will no longer have access to the CMB-S4 document and database repository, internal forums, computing resources and data.  Authorship rights will continue, or not, as prescribed in the Publication Policy bylaws.

% \item \textcolor{blue}{If independent postdocs can be members, we need to add them to the list of folks who are reviewed, especially since they would be accumulating effort toward builder status.}

\end{itemize}


\subsection{Procedure to Change Membership Policies}

The Governing Board may change Membership policies by a 2/3 majority vote of voting members. \textcolor{blue}{Or whatever voting fraction the Membership Committee decides.}  



% \subsection{Hanging questions}
% \begin{itemize}
% \item \editorial{When regular members (postdocs and grad students) move to a new institution, do they take some rights, eg data access, with them?  How do publication rights flow if postdocs move?}

% \item \editorial{What about international membership? In particular the absence of institutional buy-in may need to be predicated on there being national buy-in. This would be covered by DOE and NSF for US institutions, but would require eg. STFC buy-in for UK institutions etc. If this is the case, is there a quota of institutions and/or members based on the amount of the national buy-in? Do we count labor as a contribution? Do we accept in-kind contributions, and if so what is the mechanism for deciding both their value?}

% \item \editorial{How does the work-plan part of the senior membership application work? What if multiple people say they want to do the same thing, or no-one says they want to do something critical? What is the time-span of the written work-plan given the long duration of the project?}
% \end{itemize}




%\subsection{Requirements for Admission}

%The CMB-S4 collaboration consists of scientists, engineers, students, and others who contribute significantly to the CMB-S4 program. A rough guideline is that all collaborators should devote a substantial fraction of their research time to the CMB-S4 program over a period of several years.
%Visiting scientists and engineers (henceforth visitors), such as faculty members on sabbatical at a CMB- S4-affiliated institution, may also be admitted to the collaboration as short-term members, provided they devote a substantial portion of their research time to the collaboration.

%The process of admission to the collaboration shall accord with the membership rules applicable to their degree of seniority. A senior member of the collaboration is a member from a collaborating institution who has a permanent appointment or an appointment that under normal circumstances can be expected to be renewed indefinitely. All other members of the collaboration are regular members. Senior members are admitted to the collaboration individually. Regular members are selected by their institutions. The only distinction between senior and regular members is the method by which they are admitted to the collaboration.
%Senior members are admitted to the collaboration in one of four ways:
%\begin{enumerate}
%\item by being on the qualified elector list at the time of the initial spokesperson election,
%\item by being on the approved list of senior members when their institution was admitted to the collaboration,
%\item by being a regular member of the collaboration and receiving a position at any collaborating institution that would qualify him or her for senior membership status,
%and
%\item by individual application.
%\end{enumerate}
%A senior physicist at a collaborating institution who is not a member of the collaboration, and who wishes to join the collaboration, applies for membership in the same way as a new institution, as provided in Section 6. This procedure applies as well to visiting senior personnel applying for short-term membership.

%The membership list is distinct from the author list described in Section \ref{sec:pub}.

%\section{Admitting New Institutions}

%To apply for membership in the CMB-S4 collaboration, representatives from the institution should first confer with the spokesperson. The representatives then submit a written proposal to the IB chair at least two weeks prior to the meeting of the IB at a regularly scheduled collaboration meeting. The IB chair circulates the proposal to IB members prior to the meeting and puts consideration of the application on the meeting agenda.

%The proposal should describe the contributions that the prospective institution proposes to make to the collaboration and should include a list of the proposed senior members from the institution. The spokesperson will present the proposal to the collaboration by the prospective institution. This presentation should occur prior to the IB meeting held during the same collaboration meeting. The IB may: (1) approve the proposal and formally admit the institution to the collaboration at that meeting, (2) request further written clarifications to be considered at the next collaboration meeting, or (3) reject the proposal. The IB will attempt to make decisions on admission of new institutions by consensus. If consensus is not possible, a vote by secret ballot will be taken on admission. Admission will require a favorable vote by a super-majority of IB members present and voting.

%\section{Removal of Collaborators and Institutions, Updating Membership and Leaving an Institution}

%\subsection{Removal of Collaborators and Institutions}

%Annually the GB Chair, in consultation with the spokesperson, will identify inactive institutions, using the criterion that an active institution has performed a reasonable amount of work on CMB-S4 within the previous twelve months. The IB may remove inactive institutions from the collaboration by a super- majority vote.

%Individual collaborators who have not performed a reasonable amount of work on CMB-S4 for a period of at least twelve months, or whose behavior has been egregious and detrimental to CMB-S4, may be removed from the collaboration at an IB meeting by a super-majority vote of the IB members present and voting.

%\subsection{Updating CMB-S4 Membership Rolls}

%Each CMB-S4 IB member will be responsible for updating and verifying the collaboration membership list for his/her institution. A yearly questionnaire will be distributed by the IB chair to each institution to help keep track of all members and their status.

%\subsection{Leaving a Collaborating Institution}

%Individuals maintain their membership in the collaboration automatically for one year after they leave any collaborating institution. They can maintain a longer-term association with CMB-S4 by being ``adopted'' by a collaborating institution, with this process being approved by the IB on a case-by-case basis.

%When a person's membership in the collaboration is terminated, they will no longer have access to the CMB-S4 document and database repository, internal forums, computing resources and data.


